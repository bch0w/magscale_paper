% gjilguid2e.tex
% V2.0 released 1998 December 18
% V2.1 released 2003 October 7 -- Gregor Hutton, updated the web address for the style files.

\documentclass{gji}
\usepackage{timet}
\usepackage{amsmath}


\title[Comparisons of surface wave amplitude decays]
  {Comparisons of surface wave amplitude decays \\as measured by rotation and translation sensors}
\author[Bryant Chow]
  {Bryant Chow$^1$\thanks{Now at Victoria University of Wellington, Wellington, New Zealand}, 
  Heiner Igel$^1$, 
  Joachim Wassermann$^1$,
  Bernhard Schuberth$^1$\\
  $^1$ Department of Earth and Environmental Sciences, Ludwig-Maximilians-University Munich, Theresienstra\ss e 41, D-80333 Munich, Germany. \\E-mail: bchow@geophysik.uni-muenchen.de
  }
\date{}
\pagerange{\pageref{firstpage}--\pageref{lastpage}}
\volume{}
\pubyear{}

%\def\LaTeX{L\kern-.36em\raise.3ex\hbox{{\small A}}\kern-.15em
%    T\kern-.1667em\lower.7ex\hbox{E}\kern-.125emX}
%\def\LATeX{L\kern-.36em\raise.3ex\hbox{{\Large A}}\kern-.15em
%    T\kern-.1667em\lower.7ex\hbox{E}\kern-.125emX}
% Authors with AMS fonts and mssymb.tex can comment out the following
% line to get the correct symbol for Geophysical Journal International.
\let\leqslant=\leq

\newtheorem{theorem}{Theorem}[section]

\begin{document}

\label{firstpage}

\maketitle

\begin{summary}
Current broad-band surface wave magnitude equations relate magnitude with station-receiver distance and the vertical component of peak ground velocity, such that only contributions from the vertical component of Rayleigh waves are present. With the advent of rotational ground motion data from instruments such as ring laser gyroscopes and fiber-optic gyroscopes, it is possible to determine peak amplitudes of rotations about the vertical axis, which is theoretically only sensitive to the transverse nature of Love waves, unaffected by the horizontal component of Rayleigh waves. We use this concept to study the amplitude decay of rotations versus translations, and determine the necessity of a separate surface wave magnitude equation for Love waves. Utilizing a large database of rotation ground motion events, collected in Wettzell, Germany, we empirically define decay constants for measured observables: rotation rate, rotation, vertical velocity and transverse velocity. Results indicate that measured rotation amplitudes decay faster over distance compared to velocity amplitudes, both on vertical and transverse components. Observations are corroborated with synthetic seismograms produced on a full scale 3D global model with crustal and Moho topography models. Synthetics were created with the spectral-element method Salvus, and suggest that ...
\end{summary}

\begin{keywords}
magnitude, rotational ground motions, seismic instrumentation, amplitude decay
\end{keywords}

\section{Introduction} 
% rotations are gaining traction, large catalog of events
For over a decade, the application of ring laser gyroscope technology to the field of seismology has allowed for near-continuous, direct, measurements of rotational ground motions. An ever growing number of observations from seismic events of varying size, distance and source mechanism, has been collected in an expansive catalog of waveform recordings for both direct rotation, and collocated translation measurements.
%> we routinely compare amplitude ratios for phase velocities
Previous work on this unique waveform dataset includes phase comparisons of translations and rotations [Igel 2005] %cite
and estimations of local horizontal phase velocities. [include more previous work?] %cite
Much of this preceding work however, focuses on single events, or collections of non-earthquake sources. 

%> we want to understand how rotations decay compared to translations
%> how do we do this? we make 'magnitude scales' to quantify decay characterstics
%> these are not real magnitudes because we are geographically biased and we are not avera
%ging over hundreds of instruments like is common
In this paper, we aim to characterize and understand the differences in amplitude decay behavior of rotational and translational ground motion for a large number of events. To do this, we make use of a sizable catalog of seismic event data, measured by an observatory based ring laser gyroscope and collocated broadband translation sensor. By processing rotation and translation observations in an identical manner, we are able to directly compare processed results over a large set of event magnitudes, epicentral distances and backazimuths. We additionally seek to use this information to better understand decay characteristics of Love and Rayleigh waves.

This paper builds on work done by [Igel 2007], %cite igel 2007
where the question was posed, whether observed peak rotation amplitudes matched with expected values given by the surface wave magnitude equation. At the time, a lack of events lead to a limited study size, however a much larger number of events is now currently available. In this work, we attempt to readdress this question, while also approaching the problem from the unique perspective of deriving magnitude scales for peak amplitudes over distance, in an attempt to similarly quantify the decay characteristics of rotations and translations.

Due to the uniqueness of our instrumental setup, we were restricted to observations at a single station. As a result, Global 3D synthetic simulations were run with the wave propagation code Specfem3D Globe, such that we could compare with synthetics. Numerically generated seismograms for source-receiver pairs similar or identical to the real world observations were created. Outputs of synthetic rotations and translations were fitted with magnitude scales, and their decay characteristics were compared against observations. Real seismic event locations and source parameters were used, as well as real station locations, in order to provide the most comparable synthetic setup to our observations.

%motivation
%salvus
%finish


\subsection{Rotational Ground Motions}
Rotational ground motions induced by seismic events can currently be measured in two ways: by array methods and by direct measurement. %citation
In the former, spatial gradients are taken, for measurements in an adequately spaced array of translation sensors, in order to derive components of the strain tensor [Spudich ?]. In the latter, unique instrument designs allow for direct gradient measurements [Schreiber? Others?]. %different rotation sensors

In this study, data was recorded by the Gro\ss ring (G-ring) [Schreiber 2005, 2006], a 4x4m helium-neon ring laser gyroscope, located in Wettzell, Germany (49.144$^\circ$N, 12.87$^\circ$E). The G-ring operates on the principle of Sagnac interferometry [Stedman, 1997], which relates interference of counter propagating light beams to absolute rotation rate, through the following equation, 

\begin{equation}\label{eq:sagnac}
	\delta f = \frac{4A}{\lambda P}\mathbf{n}\cdot \mathbf{\Omega},
\end{equation}

\noindent where the constants are given by instrument area A, perimeter P and operating light wavelength $\lambda$. Equation \ref{eq:sagnac} relates an observable beat frequency $\delta f$ to rotation rate $\Omega$.

It is important to note that given stable instrument geometry and lasing, changes to the beat frequency $\delta f$, can only be introduced by changes to the inner product of the plane normal {\bfseries n} with $\mathbf{\Omega}$ (i.e. through instrumental tilt), and through externally induced rotations (i.e. the passing of seismic waves). It has been shown that changes to the inner product as produced by tilt are one to two orders of magnitude smaller than rotations produced by passing seismic waves [Igel? Schreiber?]. This gives G the very unique benefit that it is theoretically insensitive to translations, and only sensitive to externally induced rotations.

\subsubsection{Phase velocity relation}\label{phasevel}
As shown in [Igel 2005], %citation
for a simple plane wave assumption, the amplitudes of vertical rotation rate and transverse acceleration can be related through the equation $\frac{\ddot{u}_t}{\Omega_z} = -2c$, where c represents an apparent horizontal phase velocity. This relationship shows that for a sufficient distance from an event to assume a plane wave, rotations are similarly sensitive to the transverse component of translation, represented by surface horizontal waves (either SH or Love waves). It also states that waveforms of transverse acceleration and vertical rotation rate will theoretically be in phase for passing horizontal waves.
%In this paper the shortest epicentral distances considered are 2$^\circ$. A surface wave moving at a velocity of 4 km/s, with dominant period $T=20$s, covers 80 km per wavelength. As a rule of thumb, if 1$^\circ$ epicentral distance covers ~100 km distance, this gives two wavelengths as the shortest propagation distance. 

\subsubsection{Peak correlation coefficient}
Correlations are a useful measure of similarity between two signals. It has been shown previously that for collocated measurements of vertical rotation rate and transverse acceleration, high values for zero lag correlations can be retrieved in time windows centered around surface wave arrivals. %citation
Zero lag correlation coefficients are routinely computed for new events measured by the G ring [Salvermoser 2017], %citation
and are used in this study as a filtering tool to separate out events with low signal to noise ratio or dissimilar waveforms which may arise due to non-physical effects (i.e. instrumental effects). Here, the largest correlation coefficient obtained is labelled the peak correlation coefficient, and will be used here as a representation of data quality for an event.


\subsection{Magnitude scales}
Amplitude based magnitude scales provide empirically derived relationships between maximum trace amplitudes and source-receiver distances. Magnitude scales offer useful and quick estimates of relative sizes of earthquakes in a standard and easily-understandable manner. 
The International Association of Planets Seismology and Earths Interior (IASPEI) Working Group on Magnitudes has proposed a modified version of the original surface wave magnitude equation proposed by Karnik et al. and Vanek et al., %citation
which is compatible for use with modern day broadband seismic instruments [Borrmann and Bergman 2000].

In this work, we adhere strictly to the standard procedures provided by IASPEI as an outline for defining our own empirical magnitude scales. We use these derived scales as a tool for quantifying amplitude decay for different measured observables, in a standard fashion.

\subsubsection{Standard Procedures}
The Working Group on Magnitudes' standard procedures gives the revised surface wave magnitude equation for broad-band instruments as,
\begin{equation}\label{eq:mag}
	M_S^{BB} = log_{10}(V/2\pi) + B\cdot log_{10}(\Delta) + C, 
\end{equation}
where the constants $B=1.66$ and $C=0.3$ control amplitude decay and order of magnitude, respectively. The parameter V should be the maximum trace amplitude (nm s$^{-1}$) in the surface wave train, for a seismogram proportional to velocity, measured on the vertical component. 

Further criteria given by IASPEI posit that the period of the surface wave should lie within 3 s $\le$ T $\le$ 60 s, while epicentral distances should be between 2$^\circ \le \Delta \le 160^\circ$. It is further recommended that only shallow focus earthquakes should be considered, as medium to deep events are less capable of generating strong surface waves. %citation
Maximum trace amplitudes are described as one half the largest peak to adjacent trough deflection, and associated period are given as two times the temporal difference between peak and adjacent trough. All events and processing steps in this paper adhere to these guidelines.

\subsubsection{Instrumental proxies for Love and Rayleigh waves}
A standard procedure for determining surface wave magnitude scales is taking amplitudes measured on the vertical component of translation. The reason given is that vertical translation should only be sensitive to the vertical motions of Rayleigh waves, whereas vector sums of horizontal components can be influenced by both Love and Rayleigh waves. In the same vein, velocity measured on the transverse component should only show sensitivity to Love waves. This is, however not common in practice, most likely due to the necessity of rotating horizontal components to the correct backazimuth. The G-ring, which is 1) insensitive to translations and 2) proportional in phase and amplitude to transverse acceleration, should also only be sensitive to Love waves in the surface wave train, irrespective of backazimuth.  

In this study we use instruments as physical wave-filters, in order to separate phases in the surface wave train. This allows us to study the influences of Love waves and Rayleigh waves individually. By comparing the vertical and transverse components of translations, to the vertical component of rotation, we can understand, by proxy, the wave types they are sensitive to.

\subsubsection{Application of rotations to magnitude scales}
The surface wave magnitude equation is defined for peak vertical velocity amplitudes in the surface wave train. In order to give a fair comparison using derived magnitude scales, a complementary rotation parameter is necessary. In Section \ref{phasevel}, an equation is given that relates rotation rates $\Omega$ with accelerations $\ddot{u}$. It would make the most sense, then, to compare velocities $\dot{u}$ with rotations $\omega$ (by integrating both sides). However, without previous work to draw precedence from, and for completeness, we present observations of both rotations and rotation rates in this study.

\section{Event choice}
The G-ring has been continuously recording at its current resolution since May, 2009. %citation?
The time range for events used in this study spans from June, 2009 until September, 2016. An initial catalog was fetched from the Harvard Global Centroid Moment Tensor (GCMT) [Ekstr\"om et al. 2012 ?], %citation 
with events initially filtered by acceptable magnitude, source depth and epicentral distance from Wettzell, Germany. At this point we imposed the restriction that the derived 'magnitude' as given by our magnitude equations, should fall as close to the given surface wave magnitude as possible. This ensures that our derived scales do not stray too far from established scales. This meant only events with centroid moment magnitude values of $6 \le M_{\text{wc}} < 8$ (as published in the GCMT catalog), were considered;   surface wave magnitude and moment magnitude are approximately equal in this range [Shearer 2009]. %citat
Zero-lag cross correlations were taken in order to calculate peak correlation coefficients between rotation and acceleration traces (Section \ref{sec:dataproc}). Events were not considered if their peak correlation coefficient was below PCC $< 0.7$. 

These event choice criteria narrowed the catalog down to less than 500 events in the given time period. Each event was appropriately filtered and processed (Section \ref{sec:dataproc}), and waveforms were individually inspected. Waveforms that exhibited anomalous behavior (i.e. unexpected high amplitude peaks outside the surface wave train, high signal-to-noise ratio etc.) were rejected. Examples of accepted and rejected waveforms are given in Figure (put a figure here). A final event catalog of 243 events was reached.

\section{Methods}
\subsection{Data Processing}\label{sec:dataproc}
Events were processed in a similar fashion as the steps outlined in [Salvermoser 2017]. %cite johannes
Raw continuous translation data in North, East and up components, as well as vertical rotation data, was fetched based on event origin time. Instrument response correction produced translation seismograms proportional to units of velocity (nm s$^{-1}$). Epicentral distances and theoretical backazimuth values were calculated from station-receiver latitude longitude pairs, and events were separated into categories of close ($\Delta < 3^\circ$), local ($\Delta <100^\circ$) and far ($\Delta \ge 100^\circ$). Horizontal components were rotated into the transverse, radial coordinate system by the appropriate theoretical backazimuth. Measurements from ring laser gyroscope instruments do not require frequency dependent instrument correction [Sagnac?], %citation
therefore only a simple scale factor was necessary to retrieve seismograms proportional to rotation rate (nrad s$^{-1}$). Rotation rate traces were integrated to provide measurements of rotation (nrad), and transverse velocity was integrated to retrieve transverse acceleration, which was subsequently used to calculate correlations with vertical rotation rates.

A bandpass filter was applied to all traces for periods between 3 s $\le$ T $\le$ 60 s, in accordance to the IASPEI standard procedures. Peak amplitudes were chosen by finding minimum and maximum trace values and the largest associated peak or trough, respectively. The larger of the two was taken, alongside the associated arrival time and dominant period. %explain how dominant period calculated?
Theoretical considerations used to restrict search to the surface wave train proved inconsistent over a large number of events, so maximum amplitudes in the entire trace were considered. Through manual inspection, picked amplitudes that fell outside the surface wave train were rejected. % could go back and justify this better by stating the order of magnitude differences weren't that big since were only looking at log scales

\subsubsection{Zero-lag correlations}
To calculate peak correlations, traces of transverse acceleration and vertical rotation rate were segmented into small time windows based on the event-station distance. In each time window, a zero-lag correlation was performed, and a single value of correlation produced. From the entire trace, the max value was taken to represent the peak correlation coefficient. For most waveforms, the peak correlation coefficient lie in the surface wave train. As mentioned previously, these peak correlation values are used extensively as a ranking system for events, providing a quickly attainable measure of waveform quality.

\subsection{Curve fitting}
To quantify amplitude decay, magnitude scale coefficients were fit to the data using a simple linear regression. Equation \ref{eq:mag} represents a relationship between magnitude and amplitude for a single event. A collection of $n$ events can be represented in the form, 
\begin{equation}
	\begin{pmatrix}
		log_{10}(\Delta_{1}) & 1 \\
		log_{10}(\Delta_{2}) & 1 \\
		\vdots  & \vdots \\
		log_{10}(\Delta_{n}) & 1 
	\end{pmatrix}
	\begin{pmatrix}
		{B}\\
		{C}
	\end{pmatrix}
	=
	\begin{pmatrix}
		M_{\text{wc}_1} - log_{10}({V_1}/{2\pi})_{\text{max}} \\
		M_{\text{wc}_2} - log_{10}({V_2}/{2\pi})_{\text{max}} \\
		\vdots  \\
		M_{\text{wc}_n} - log_{10}({V_n}/{2\pi})_{\text{max}}
	\end{pmatrix},
	\label{eq:linearreg}
\end{equation}

\noindent which can be further condensed to the form, $\mathbf{Gm = d}$. The unknowns B and C are represented in the vector {\bfseries m}, and can be solved for through the normal equation $\mathbf{m} = \mathbf{(G}^{T}\mathbf{G})^{-1}\mathbf{G}^T\mathbf{d}$.
By determining values of B and C, we create an empirical magnitude scale that best describes the amplitude decay behavior of our events. In Equation \ref{eq:linearreg}, we impose that our derived magnitude value should be as close to an events given moment magnitude as possible, by setting $M_S^{BB}$ equal to the value of $M_\text{wc}$ retrieved from our event catalog. 

95\% confidence intervals were constructed for each parameter of the vector $\mathbf{m}$. These were calculated with the variance of estimates of the $j$th parameter of $\mathbf{m}$ by the equation $\hat{m}_j \pm c \sqrt{\hat{var}(\hat{m}_j)}$, where the value of c is given as 1.96.

\section{Synthetic seismograms}
Due to the unique instrumental setup of the G-ring, there are currently no other instruments with as much temporal coverage to draw data from for comparisons. Single events are available from other rotation instruments [Donner 2017 PFO?], however these are insufficient for the purposes of this paper. One possibility for gathering more observations would be through array derived rotations as a substitute for direct rotation measurements [Spudich ?]; a lack of long-term arrays with the optimal station spacing, however, limits this option. In lieu of this, we turn to waveform modeling to generate synthetic seismograms, with which we can recreate our experimental setup and provide a comparable set of observations.

The wave propagation code Specfem3D Globe was employed to generate synthetic observations. A realistic global model featuring 3D crust and mantle models was used, and the simulation featured effects that might potentially influence surface waves at the periods of interest. These effects include: ocean loading, ellipticity, topography, self gravitation, Earth's rotation and 1-D attenuation. Event locations and moment tensors were taken from a handful of real seismic events present in the observation catalogs. Events were chosen based on data quality, to provide the highest quality comparisons of observations and synthetics, as well as event location and depth, so as to provide a varied distribution of source-receiver pairings. Table \ref{tab:syn_events} provides more information on the chosen events. 

A simulation corner frequency was set to 10 s and simulations were run for (how much synthetic time?). To generate a larger swath of observations, many stations, both real and synthetic, were included. Real stations included in every event were set as the locations of the Global Seismic Network (GSN), with inclusion of the G-ring location in Wettzell, as well as the observatory station F\"urstenfeldbruck. Synthetic station locations were used to boost the resolution at small epicentral distances. Data was outputted to roughly 50,000 stations equally spaced around the surface of the Earth. For each event, a small radius of stations surrounding the epicenter was chosen, and these data included in the calculation of the magnitude scale.

\begin{table*}
\begin{minipage}{150mm}
	\begin{center}
		\begin{tabular}{ |c|c|c|c|c|c|c|c| } 
		\bf{Date} & \bf{Time (UTC)} & \bf{Lat($^\circ$)} & \bf{Lon($^\circ)$} & \bf{Depth(km)} & \bf{M$_{\text{wc}}$} &\bf{Region} &\bf{Peak Corr. Coeff.}\\ \hline
	2010-07-18 & 13:34:59 & -5.93 & 150.59 & 35.0 & 7.32 & New Britain Region, P.N.G. & 0.98\\
	2011-09-16 & 19:26:41 & 40.27 & 142.78 & 35.0 & 6.67 & Off East Coast Of Honshu, Japan & 0.99\\
	2013-01-05 & 08:58:19 & 55.39 & -134.65 & 10.0 & 7.53 & Southeastern Alaska & 0.95\\
	2013-04-16 & 10:44:20 & 28.03 & 62.0 & 80.0 & 7.74 & Southern Iran & 0.98\\
	2013-04-19 & 19:58:40 & 49.97 & 157.65 & 15.0 & 6.06 & East Of Kuril Islands & 0.99\\
	2015-02-13 & 18:59:12 & 52.65 & -31.9 & 16.7 & 7.07 & Reykjanes Ridge & 0.99\\
	2015-04-25 & 06:11:26 & 28.15 & 84.71 & 15.0 & 7.88 & Nepal & 0.99\\
	2015-09-13 & 08:14:12 & 25.14 & -109.43 & 10.0 & 6.6 & Gulf Of California & 0.98\\
	2015-09-16 & 23:18:41 & -31.56 & -71.43 & 28.4 & 7.1 & Near Coast Of Central Chile & 0.99\\
	2016-01-25 & 04:22:02 & 35.65 & -3.68 & 12.0 & 6.38 & Strait Of Gibraltar & 0.99\\
		\end{tabular}
    		\caption{List of events used as synthetic sources in Specfem3D. Peak correlations are used as a measure of waveform quality, and only events with the highest values were used, in order to provide the best comparisons of synthetics with observations. Events were also chosen based on a diverse coverage of magnitudes and epicentral distances from the ring laser stationed in Wettzell, Germany. Event information taken from the GCMT catalog.}
		\label{tab:syn_events}
	\end{center}
	\end{minipage}
\end{table*}

The direct outputs of Specfem3D were adjusted to produce displacement in the transverse, radial and vertical components, by rotation with respect to the theoretical backazimuth. Direct rotation in the same coordinate system was also produced as output. A workflow identical to that used for observations was employed to calculate peak trace amplitudes, and a magnitude equation was fit to the data for comparison.

\subsection{Comparative Study}
The recent availability of Salvus, a suite of software developed at ETH Zurich, aimed at full waveform modeling and inversion, it was possible to make a differential study with respect to the outputs of Specfem. To do so we chose one event from those listed in Table \ref{tab:syn_events}, specifically the M$_{wc}$ 2011 Japan event, as a case study for comparisons. In Salvus we again generated a realistic mesh with 3D crustal structures, as well as topography and effects including (???). With the same station locations as Specfem, we were able to generate seismograms and a single event magnitude scale, with which to compare against our observations and synthetics.

\section{Results}
\subsection{Derived magnitude scales}
For rotations and translations, decay characteristics were derived by solving for constants B and C in the magnitude equation. These values are presented in a Table \ref{tab:scales}. Since the constant C is dependent on the order of magnitude of the units, it is difficult to compare this between the different observation types. The value of B, however, controls the decay of amplitude with distance, and allows us to fairly compare rotations and translations. To check the results, the same processing was performed on observations taken at the geophysical observatory in F\"urstenfeldbruck, Germany (48.163$^\circ$N, 11.275$^\circ$E), which is located roughly 200 km to the south-west of Wettzell. Though a smaller subset of events was used due to data availability, the results confirmed those given by Wettzell by returning derived constants B and C within 1\%. To reduce clutter of results, however, these are not shown here (show in appendix?)

Consulting Table \ref{tab:scales}, rotation has the closest decay constant B to the surface wave magnitude equation. Rotation rate exhbiits even higher 


\begin{table*}
\begin{minipage}{115mm}
	\begin{center}
		\begin{tabular}{ |l|c|c|c|c| } 
		        \bf{Scale} & \bf{Label} & \bf{B} & \bf{C}  & \bf{Wave}\\ \hline
	IASPEI & $M_{S}^{BB}$ & 1.66 & 0.3  & Rayleigh \\ \hline
        Rotation  & $M^{RLAS}_{RT}$ & 1.557 $\pm$ 0.295 & 4.186 $\pm$ 0.569  & Love \\ \hline
	Rotation Rate & $M^{RLAS}_{RR}$ & 1.823 $\pm$ 0.303 & 4.113 $\pm$ 0.586  & Love\\ \hline 
        Transverse Velocity & $M^{WET}_T$ & 1.45 $\pm$ 0.27 & 0.527 $\pm$ 0.521 & Love \\ \hline
        Vertical Velocity  & $M^{WET}_Z$ & 1.084 $\pm$ 0.264 & 1.093 $\pm$ 0.511  & Rayleigh \\ \hline
		\end{tabular}
		
    		\caption{Magnitude scales and derived constants with 95\% confidence intervals for observations at station RLAS/WET, for equations of the form $M = log_{10}(V/2\pi) + B\cdot log_{10}(\Delta) + C$. The final column gives consideration to the wave type that each instrument component should provide a proxy for.}
		\label{tab:scales}
	\end{center}
	\end{minipage}
\end{table*}

\begin{table*}
\begin{minipage}{115mm}
	\begin{center}
		\begin{tabular}{ |l|c|c|c|c| } 
		        \bf{Scale} & \bf{Label} & \bf{B} & \bf{C}  & \bf{Wave}\\ \hline
        Synthetic Rotation  & $M^{SYN}_{RT}$ & 1.557 $\pm$ 0.295 & 4.186 $\pm$ 0.569  & Love \\ \hline
	Synthetic Rotation Rate & $M^{SYN}_{RR}$ & 1.823 $\pm$ 0.303 & 4.113 $\pm$ 0.586  & Love\\ \hline 
        Synthetic Transverse Velocity & $M^{SYN}_T$ & 1.45 $\pm$ 0.27 & 0.527 $\pm$ 0.521 & Love \\ \hline
        Synthetic Vertical Velocity  & $M^{SYN}_Z$ & 1.084 $\pm$ 0.264 & 1.093 $\pm$ 0.511  & Rayleigh \\ \hline
		\end{tabular}
		
    		\caption{Synthetic magnitude scales and derived constants with 95\% confidence intervals, for equations of the form $M = log_{10}(V/2\pi) + B\cdot log_{10}(\Delta) + C$. The final column gives consideration to the wave type that each instrument component should provide a proxy for.}
		\label{tab:syn_scales}
	\end{center}
	\end{minipage}
\end{table*}

\subsection{Expected amplitudes}
One powerful use of magnitude scales is to provide a magnitude value, and predict the expected value of ground velocity at a certain distance. This is useful, for example, in seismic hazard analysis, where one can estimate the order of magnitude of ground motion at a location some distance away from an event, given a certain event size. In this way, we can try to determine the predicted ground rotation for a specific magnitude. This will be quite useful for the future of field-based rotational seismology, where resolution of field-deployable sensors is quite often much lower than those achieved by observatory based instruments, which have the benefit of large construction size and spatial stability. At the time of writing, field deployable rotation sensors have a noise floor of about (???) [Cite ???]. With our given magnitude equation, we can predict the farthest distance necessary to obtain a useable signal to noise ratio of vertical rotation or rotation rate. For example, given a magnitude 6 event, according to our observations, one must be within (???) epicentral distance to achieve some signal. One hopes this proves a useful tool in future field work, perhaps for catching aftershocks at a reasonable distance.

\subsection{Synthetic results}
Synthesize.

\section{Discussions and conclusions}
Long term continuous recordings of collocated rotation and translation has now provided a superb catalog of seismic events useable in comparisons of direct translation measurements and direct measurements of spatial gradients. Phase information was the logical starting point, but now the measurement of amplitudes proves useful in understanding both subsurface velocity structures as well as amplitude decays of surface waves. 

The results of this work show the consistency of rotation measurements for a large observatory based ring laser gyroscope, and allow very accurate and stable comparisons of rotations and translations. Since events were manually chosen to exhibit the best characteristics for their waveforms, we can be sure that the derived decay characteristics that are shown here also are stable, albeit for teleseismic waves. The results at face value show that observed rotations for the event catalog used decay faster than observed translations for the same events, and that rotation rate decays faster than the rotation, though this is to be excepted as we are simply looking at time derivative information. What is most interesting is how the rotation and rotation rate decays match much closer to the globally averaged surface wave magnitude as compared to the translation measurements which it is based off. Understandably a global average should be taken to give a more reasonable value for the decay. It should be pointed out, however, that even without global averaging, it is exciting to see the characteristics of this new observable, and by taking comparisons of direct measurements, we are gaining more understanding of the behaviors of this observable.


\label{lastpage}


\end{document}
